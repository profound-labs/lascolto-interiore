\chapter{L'Ascolto Interiore}

Ci sono alcuni argomenti che i praticanti della meditazione buddhista
conoscono bene: ``la consapevolezza del respiro'', in cui ci si
focalizza sul ritmo del respiro, ``la meditazione camminata'', che ruota
intorno alla sensazione dei passi mentre si cammina su e giù lungo un
percorso prefissato, la ripetizione interna di un mantra, come
``Bud-dho''. Tutte queste pratiche sono concepite per ancorare
l'attenzione nella presenza di questo esatto momento, di questa realtà
presente.

Insieme a questi metodi più conosciuti ce ne sono molti altri che
possono essere utilizzati allo stesso scopo. Uno di questi è noto come
``ascolto interiore'' o ``meditazione sul suono interiore'', o ancora,
in sanscrito, ``\emph{nada yoga}''. Questi termini si riferiscono tutti
al metodo in cui si presta attenzione a quello che è stato chiamato ``il
suono del silenzio'', o ``il suono-nada''. ``Nada'' è la parola
sanscrita per ``suono'', come pure la parola spagnola per ``nulla''. Una
coincidenza interessante e - del tutto casualmente - significativa.

\section{Il Suono Interiore e Come Trovarlo}

Il suono-nada è un tono interno acuto e squillante. Se rivolgete la
vostra attenzione all'udito e ascoltate attentamente i suoni intorno a
voi udirete un suono continuo e acuto, come un rumore bianco - senza
inizio, senza fine - che brilla sullo sfondo.

Provate a discernere quel suono, e dirigetegli la vostra attenzione. Per
il momento non c'è bisogno di formulare teorie in proposito domandandovi
che cosa potrebbe essere esattamente, semplicemente rivolgetegli la
vostra attenzione. Osservate se riuscite a percepire quella gentile
vibrazione interna.

Se siete in grado di udire quel suono interiore potete usare il semplice
atto dell'ascolto come un'altra forma di pratica meditativa,
utilizzandolo, proprio come il respiro, quale oggetto di consapevolezza.
Semplicemente portate l'attenzione su quel suono, e lasciate che riempia
tutta intera la sfera della vostra consapevolezza.

\section{Le Due Dimensioni Del Samādhi}

La concentrazione meditativa, samādhi, può essere descritta come ``un
oggetto mentale che riempie la consapevolezza per un periodo di tempo'',
o anche ``il fissarsi della mente su un singolo oggetto''. Quindi il
samādhi è concentrazione unificata in un punto, tuttavia questa
focalizzazione dell'attenzione può funzionare in due modi diversi.
Innanzitutto, possiamo pensarla come ``il punto che esclude'', che si
chiude, cioè, intorno a un singolo oggetto escludendo tutto il resto.
Questa prima modalità è limitata, circoscritta, come quando si concentra
il fascio di luce di una torcia regolabile. Questa è la base di
\emph{samatha}, vale a dire calma o tranquillità.

La seconda modalità può essere meglio descritta come ``il punto che
include'', si tratta, cioè, di una consapevolezza espansa, che considera
l'interezza del momento presente quale oggetto di meditazione. Al
``singolo punto'' viene consentito di espandersi fino a includere tutte
le forme dell'esperienza presente. Un po' come quando si usa la modalità
ad ampio raggio della stessa torcia regolabile, tutti i vari oggetti del
presente sono inclusi nella luce della consapevolezza, piuttosto che
esservi un solo luogo fortemente illuminato. Questa è la base della
vipassanā, o visione profonda.

Uno dei grandi benefici della meditazione sul suono interiore è che può
servire facilmente di supporto a entrambi i tipi di samādhi, tanto il
punto che esclude quanto quello che include.

\section{Nada Come Supporto Della Tranquillità -- Samatha}

Possiamo prendere il suono interiore come oggetto primario di
attenzione, lasciandogli occupare l'intero spazio di ciò che viene
conosciuto. In piena consapevolezza lasciamo che tutto il resto - le
sensazioni nel corpo, i rumori che udiamo, i pensieri che possono
sorgere - rimangano alla periferia, ai margini della nostra sfera di
interesse. Consentiamo invece al suono interiore di occupare
completamente il fuoco della nostra attenzione, lo spazio di questa
consapevolezza, diventando così il sostegno diretto per l'insorgere di
samatha, la tranquillità. Proprio come il respiro, possiamo usare il
suono per governare la nostra attenzione e come oggetto singolo che
aiuti a radicare concentrazione e stabilità, continuità e unità di
attenzione nel presente.

\section{Nada Come Supporto Della Visione Profonda -- Vipassanā}

Se ci focalizziamo sul suono interiore per un periodo di tempo
sufficiente ad apportare una qualità di fermezza per cui la mente dimora
agevolmente nel presente possiamo poi consentire al suono di retrocedere
in secondo piano. In questo modo il suono stesso diventa come uno
schermo sul quale tutti gli altri suoni, le sensazioni fisiche, gli
stati d'animo e le idee vengono proiettati, uno schermo sul quale si
manifesta il film della nostra esperienza con i suoi schemi ripetitivi.
E, grazie alla sua semplicità e uniformità, è un ottimo schermo: non
interferisce con gli altri oggetti che sorgono, né li confonde, eppure è
chiaramente presente. È come se un film venisse proiettato su uno
schermo leggermente chiazzato o comunque con delle peculiarità, così
che, se prestate attenzione, vi rendete conto che c'è uno schermo sul
quale la luce è diretta. In tal modo vi rammenterete: ``Questo è solo un
film. È soltanto una proiezione. Questa non è la realtà''.

Possiamo così lasciare che il suono sia una semplice presenza sullo
sfondo, e proprio quella presenza serve da promemoria, aiutandoci a
ricordare: ``Ah, questi sono solo \emph{saṅkhāra}, formazioni mentali
che sorgono e cessano. Tutte le formazioni sono insoddisfacenti:
\emph{sabbe saṅkhāra dukkha}. Se qualcosa si forma, se è un `esso', se è
una `cosa', c'è una qualità di \emph{dukkha} nella sua stessa
impermanenza, nella sua stessa natura di `cosa'. Perciò non ti
attaccare, non rimanerci intrappolato, non ti identificare, non te ne
appropriare, prendendola per chi o cosa siamo. Lascia andare''.

Facendo così la presenza del suono può facilitare il modo con il quale
ogni \emph{saṅkhāra} viene visto come vuoto e senza un soggetto di
appartenenza, che si tratti di una sensazione fisica, un oggetto visivo,
un sapore, un odore, uno stato d'animo, una condizione raffinata di
felicità, o qualsiasi altra cosa, aiutando a sostenere un'oggettività,
una consapevolezza, una partecipazione nel presente prive di
coinvolgimenti.

C'è il flusso delle sensazioni, il peso del corpo, il tocco dei nostri
abiti, l'alternarsi degli stati d'animo, la stanchezza, il dubbio, la
comprensione, l'ispirazione, qualsiasi cosa. In tutto questo il suono
aiuta a sostenere una chiara oggettività tra le diverse modalità di
stati d'animo, sensazioni e pensieri, permettendo al cuore di dimorare
in una qualità di consapevolezza vigile. Ed è proprio quella
consapevolezza che conosce e riceve il flusso dell'esperienza -
conoscendola, lasciandola andare, riconoscendo la sua trasparenza, la
sua vacuità, la sua insostanzialità.

Il suono interiore continua sullo sfondo, ricordandoci che tutto è
Dhamma, tutto è un attributo della natura che viene e va, in mutamento,
tutto qui. Questa è una verità che magari intuiamo da molto tempo, ma
che dimentichiamo a causa della confusione che deriva dall'attaccamento
alla nostra personalità, ai nostri ricordi, ai nostri stati d'animo e
pensieri, ai malesseri del corpo, alle nostre brame.

Lo stress che deriva dall'essere aggrappati sin dalla nascita alle
esperienze, un giorno dopo l'altro, mantiene l'attenzione in uno stato
di confusione, di trance, di disorientamento. Ciò nonostante, possiamo
usare la presenza del suono-nada per aiutarci a spezzare la trance e
porre fine a quell'incantesimo, consentendoci di conoscere il flusso di
sensazioni e stati d'animo per ciò che sono, modalità della natura che
vengono, vanno e si trasformano, compiendo il loro corso. Esse non sono
chi o cosa siamo, e non possono mai veramente soddisfarci, né, quando
viste con comprensione profonda, deluderci.

\section{Nada Favorisce L'Ascolto Dei Propri Pensieri}

Man mano che l'ascolto interiore viene sviluppato come aspetto della
meditazione formale ci si accorge come prestando attenzione a un oggetto
uditivo si impari più facilmente ad ascoltare oggettivamente i propri
pensieri e stati d'animo.

Per molti versi il chiacchiericcio della nostra mente pensante non ha
maggior significato del ronzio scintillante del suono-nada. Si tratta
semplicemente delle vibrazioni della mente pensante che si compongono in
schemi concettuali, tutto qui\ldots{} solo un lungo, continuo flusso
mormorante di vibrazioni. Possiamo dunque imparare ad ascoltare il
nostro stesso pensiero come ascolteremmo lo scroscio di un ruscello, lo
zampillio di una fontana o il canto corale di uno stormo di uccelli, con
lo stesso tipo di libertà dal coinvolgimento o dall'identificazione. È
solo il torrente mormorante della mente, tutto qui. Niente di che, nulla
di cui appassionarsi o da cui essere disturbati.

Ora, questo è abbastanza facile a dirsi, ma noi tendiamo a innamorarci
delle nostre narrative, non è forse così? Ci piacciono davvero molto,
particolarmente quelle che ci riguardano: il bene che abbiamo fatto, il
male che abbiamo fatto, le cose memorabili, toccanti, deplorevoli, e ciò
che vogliamo fare, ciò che speriamo di fare, ciò che temiamo ci accadrà,
ciò che gli altri pensano di noi.

Questi schemi mentali così amati sono tutti manifestazioni dell'elemento
``io'', abitudini di pensiero in termini di ``io'', ``me'' ``mio'' di
tutta una vita. In pali sono chiamati \emph{ahamkāra}, ``che producono
l'idea dell'io'' e \emph{mamamkāra}, ``che producono l'idea di mio'' e
sono gli attributi chiave della visione egoica. Sono queste le abitudini
che attraggono più efficacemente e ripetutamente la nostra attenzione
nella sfera dei concetti, i quali poi trascinano lontano la mente. Se
una storia ha ``me'' al suo interno, tende a essere molto più
interessante di altre vicende più remote. Questo è estremamente
naturale, è un'abitudine fondamentale di tutti noi.

Allo stesso modo, una gran parte della meditazione di visione profonda,
lo sviluppo della vipassanā, riguarda proprio questo: imparare a
riconoscere le abitudini di ``costruzione dell'io'' e ``costruzione del
mio'' all'interno dei pensieri che ci attraversano. Ahamkāra significa
letteralmente ``fatto dal senso dell'io'', mentre mamamkāra significa
``fatto dal senso del mio''; è retta comprensione riconoscere quelle
abitudini e non farsi catturare dalla narrativa, vederne la vacuità, la
trasparenza, e lasciarla andare.

\section{Nada, Vacuità e Talità}

La maggior parte dei praticanti buddhisti, indipendentemente dalla
propria tradizione, ha familiarità con quelle che sono conosciute come
``le tre caratteristiche dell'esistenza'', \emph{anicca}, \emph{dukkha},
e \emph{anattā} (impermanenza, insoddisfazione e non-sé). Queste sono le
qualità universali di tutte le esperienze che sorgono e passano, e
riconoscere la loro presenza è l'aspetto più attivo della meditazione
vipassanā.

Ci sono, però, altre caratteristiche universali dell'esistenza che
possono essere analogamente impiegate per aiutare a liberare il cuore da
ogni costrizione, peso e stress. Due di queste caratteristiche, che in
qualche modo operano in coppia, prendono il nome di \emph{suññatā} e
\emph{tathatā}, rispettivamente vacuità e talità. Il termine ``vacuità''
deriva dal dire ``No'' al mondo fenomenico: ``Non crederò a questo. È
privo di sostanza, vuoto, vacuo, non del tutto reale.''

La ``talità'' è una qualità che si accompagna alla ``vacuità'', allo
stesso modo in cui la mano destra si accompagna alla sinistra.
Contrariamente però alla sua compagna, la sua natura deriva dal dire
``Sì'' all'universo. Può darsi che qui non ci sia nulla di solido,
separato o individuale - che si tratti di un pensiero, una giunchiglia o
una montagna - eppure \emph{qualcosa} c'è, alla base c'è una Realtà
Ultima che permea, abbraccia e costituisce ogni cosa. La parola
``talità'' esprime dunque un apprezzamento della vera natura di quella
Realtà, e la sua realizzazione può essere caratterizzata dal fatto di
conoscere e incarnare la presenza dell'Incondizionato, di Ciò Che Non
Muore o \emph{Amata-dhamma}.

Quando nel Canone in Pali, i testi scritturali del buddhismo
meridionale, si parla di vacuità, questa in genere ha il significato di
``privo di un sé e di ciò che appartiene a un sé'', ma si riferisce
anche a un'insostanzialità degli oggetti. Sviluppare l'abilità
dell'ascolto interiore e dello stare con il suono-nada può incrementare
enormemente la propria capacità di realizzare entrambi questi tipi di
vacuità: la vacuità sia del soggetto che dell'oggetto, sia di se stessi
che dell'altro.

Quando si è instaurato l'ascolto del suono-nada in modo ragionevolmente
fermo e stabile, cosicché il suo iridescente tono argentino sia una
presenza costante, questo facilita il riconoscimento
dell'insostanzialità di tutte le attitudini e i pensieri basati
sull'``io-me-mio'', così come descritto prima. È come una luce brillante
grazie alla quale possiamo vedere chiaramente l'inconsistenza delle
bolle di sapone che ci fluttuano accanto.

In modo simile, per tutti gli oggetti mentali che sperimentiamo - come
le cose che vediamo, udiamo, odoriamo, gustiamo e tocchiamo, e tutti i
ricordi, i progetti, gli stati d'animo e le idee che sorgono nelle
nostre menti - la presenza del suono-nada aiuta a illuminare la
trasparenza di queste modalità della coscienza. Secondo le parole del
Buddha:

\begin{quote}
La forma materiale è come una massa di schiuma,\\
la sensazione una bolla d'acqua;\\
la percezione è solo un miraggio,\\
le volizioni come un fusto di banano,\\
la coscienza un gioco d'illusione:\\
così dice il Parente del Sole.

Comunque lo si osservi\\
o con attenzione lo si investighi,\\
tutto appare vuoto e cavo\\
quando si contempla a fondo.\hfill [\emph{SN 22.95}]
\end{quote}

Il suono-nada può aiutarvi a ricordare anche la talità di ogni
esperienza. Per quanto queste qualità possano sembrare contraddittorie,
è più corretto dire che sono complementari. Quando si presta ferma
attenzione al suono del silenzio e gli si consente di colmare lo spazio
interiore della consapevolezza, la sua qualità energetica, unita alla
ricchezza senza forma della sua presenza, è un forte richiamo intuitivo
alla qualità della talità. È quasi come se (almeno per i nativi di
lingua inglese) il suono interiore esprima un infinito
``issssssssss\ldots{}'' (è) o ``thussssssss\ldots{}'' (tale), per
esortare a tornare alla realtà (che ``è tale'').

La talità è, per definizione, difficile da mettere a fuoco
concettualmente. Ha una qualità intrinsecamente sfuggente che potrebbe
farla sembrare vaga o irreale ma, paradossalmente, questa è una parte
necessaria del suo significato. È significativo che la parola stessa che
il Buddha ha coniato per riferirsi a se stesso fosse \emph{Tathāgata},
che significa ``Colui che è arrivato alla talità'' oppure ``Colui che è
andato alla talità'', a seconda dell'interpretazione. Così sebbene la
parola ``talità'' possa recare in sé una sfumatura di intangibilità,
questa è intenzionale e occorre riconoscerla come veicolo di una realtà
fondamentale.

Si potrebbe fare un paragone con il mondo della matematica e il concetto
di radice quadrata di meno uno. Nel mondo dei numeri reali non c'è
nessun numero intero che moltiplicato per se stesso dia -1. Se,
tuttavia, un tale numero esistesse davvero, allora si aprirebbe ogni
sorta di possibilità stimolante, come fu scoperto nell'antichità e poi
sviluppato dai matematici nel diciottesimo secolo.

È interessante che questo numero, sebbene non esista nel mondo reale e
abbia soltanto uno status immaginario, riesca lo stesso a essere
indispensabile per la costruzione di oscillatori a fasi alterne (usati
nell'ingegneria sonora), e sia molto usato in settori quali la grafica
computerizzata, la robotica, l'elaborazione dei segnali, le simulazioni
computerizzate e la meccanica orbitale.

Questo è per dire che, pur essendo inafferrabile, proprio come la
talità, esso ha una presenza chiara e dimostrabile nel mondo reale.\cite{imaginary}

\section{Nada e ``Atammayatā'' -- Vedere Il Mondo Nella Mente}

Una terza e ancor più sottile caratteristica dell'esistenza prende il
nome di ``\emph{atammayatā}''. La parola letteralmente significa ``non
fatto di quello''.

Quando prendiamo in considerazione le qualità di vacuità e talità,
nonostante la presunzione dell'idea ``io sono'', \emph{asmi-māna}, sia
già stata realizzata e compresa, possono ancora rimanere delle sottili
tracce di attaccamento; attaccamento all'idea di un mondo oggettivo
conosciuto da un conoscere soggettivo, pur non essendo distinguibile
alcun senso di un ``io''. Ci può essere la sensazione di ``questo'', che
sta conoscendo ``quello'', e di un ``Sì'' nei suoi confronti, nel caso
della talità, oppure ``No'' nel caso della vacuità.

Atammayatā è la fine di tutta quella dimensione, ed esprime la profonda
intuizione: ``Non c'è alcun `quello.'\thinspace ,'' È l'autentico crollo sia
dell'illusione di separatezza di soggetto e oggetto, che della
discriminazione tra i fenomeni in quanto, in qualche modo,
sostanzialmente diversi l'uno dall'altro.

Un mezzo per sviluppare questa comprensione profonda a livello pratico è
combinare, come ora vedremo, l'ascolto del suono-nada con una semplice
riflessione.

Noi tendiamo a pensare che la mente sia nel corpo. In realtà ci
sbagliamo: piuttosto è il corpo che è nella mente. Tutto ciò che
sappiamo del corpo, adesso e in passato, è stato conosciuto attraverso
l'operato della nostra mente. Questo non vuol dire che non esista un
mondo fisico, ma ciò che possiamo dire per certo è che l'esperienza del
corpo, e l'esperienza del mondo, accadono all'interno della nostra
mente.

Accade tutto qui. E quando questo essere-qui viene davvero riconosciuto
e ci risvegliamo a questa verità, l'alterità del mondo, la sua
separatezza cessano. Quando ci rendiamo conto che conteniamo il mondo
intero al nostro interno, allora il suo essere oggetto, il suo essere
altro sono tenuti sotto controllo, e siamo più capaci di riconoscerne la
vera natura.

Se ci si focalizza sul suono interiore, riflettendo e semplicemente
ricordando: ``Il mondo è nella mia mente. Il mio corpo e il mondo sono
qui in questo spazio di consapevolezza, permeato dal suono del
silenzio'', questo darà luogo eventualmente a un cambiamento di visione.
Considerando le cose in questo modo, ci si accorgerà improvvisamente che
il corpo, la mente e tutto il mondo sono arrivati a una risoluzione; c'è
una realizzazione di perfezione ordinata. Il mondo è in equilibrio
dentro quel cuore di vibrante silenzio.

Atammayatā è la qualità in voi che sa: ``Non c'è nessun \emph{quello}.
C'è solo \emph{questo}.'' Allora, quando la verità di ciò viene
compresa, persino il ``questo'' e il ``qui'' diventano privi di
significato. La presenza del suono-nada vi aiuta a realizzare e
mantenere tale prospettiva. In questo modo la mente perde lentamente la
sua abitudine di volersi sempre volgere verso l'esterno, lasciandosi
catturare dalle tendenze esteriori, \emph{āsava}, perdendosi così in
preoccupazioni mondane. Svilupperete un tranquillo controllo, una
compostezza interiore e una libertà dalle compulsioni che assalgono il
cuore così facilmente bloccandoci e imprigionandoci.

Atammayatā aiuta il cuore a liberarsi dalle più sottili abitudini
all'irrequietezza, e ad acquietare i riverberi delle nostre convinzioni
più radicate e illusorie circa la dualità di soggetto e oggetto. Questa
pacificazione porta il cuore a una intuizione radicale: c'è solo
l'interezza del Dhamma, completa spaziosità e realizzazione. Le
apparenti dualità di questo e quello, soggetto e oggetto vengono viste
come fondamentalmente prive di senso.

\section{Nada Include Attività e Impegno}

Una volta che siete riusciti a sviluppare un'attenzione stabile sul
suono-nada durante la pratica della meditazione formale seduta, potete
estenderla fino a farla diventare anche parte della meditazione
camminata. Vi accorgerete che sebbene gli occhi siano aperti e il corpo
stia camminando con regolarità avanti e indietro tra le due estremità
del vostro sentiero di meditazione camminata, potete ancora udire il
suono-nada che abbraccia tutto. È sempre lì, saldamente sullo sfondo,
che permea tutta l'esperienza e vi aiuta a ricordare che tutto questo è
conosciuto all'interno della sfera della vostra consapevolezza. Il corpo
e il mondo sono in effetti all'interno della mente.

Man mano che diventate sempre più capaci di mantenere l'attenzione sul
suono del silenzio in presenza di questi diversi oggetti dei sensi,
scoprirete che lo si può seguire in quasi tutte le situazioni. La vostra
consapevolezza diventerà più solida.

Mentre camminate per strada, giocate con i vostri bambini, partecipate a
una riunione di lavoro, pranzate o aspettate in coda, mentre siete
seduti in aereo, parlate con i vostri amici, guardate la televisione,
scrivete un articolo o siete in visita da vostra madre, persino nel bel
mezzo di un'attività turbolenta o in presenza di rumori intensi, come un
traffico pesante, una motosega azionata nelle vicinanze o un martello
pneumatico, se prestate ascolto il suono è sempre lì. Perciò lo possiamo
sempre usare come supporto alla presenza mentale e alla chiara
consapevolezza.

Inoltre, se lo utilizziamo per ricordarci di mantenere le cose così in
prospettiva, il suono ci aiuta a relazionarci all'attività in questione
con maggiore sensibilità. In qualche modo sembra che non divida la
nostra attenzione quanto piuttosto la accresca. In aggiunta, prestandovi
attenzione nel bel mezzo di attività e impegni, ci consente di osservare
la situazione con il quadro più sgombro dalla preoccupazione di se
stessi.

Vi state dando la possibilità di rispondere in modo consapevole alle
innumerevoli circostanze ed esperienze della vita, in accordo con le
leggi della natura, invece di reagire ciecamente spinti dall'abitudine e
dalla compulsione. Potete liberarvi dal ciclo infinito della brama e del
rimorso nel quale la maggior parte di noi si trova intrappolata.

\section{Nada e Lo Sviluppo Della Compassione}

Oltre ad aiutare ad affrancare il cuore da tali tendenze ostruttive e a
sostenere qualità salutari nel bel mezzo di attività e impegni, la
presenza del suono-nada può essere anche usata per far sorgere e
mantenere gentilezza e compassione. Quando riflettiamo su come di solito
accogliamo e ci relazioniamo con il mondo, queste sono le qualità più
preziose e utili da coltivare.

È significativo che nella tradizione del buddhismo del Nord, il
Bodhisattva Guan Yin, o Avalokiteśvara, svolga il ruolo di
personificazione della compassione. Il nome significa ``Colei Che Presta
Ascolto ai Suoni del Mondo'', e in tal senso è una potente indicazione
di quali siano le radici della vera compassione. Sebbene si possa
ridurre la compassione al ``fare cose utili per gli esseri che stanno
soffrendo'', questo nome (e in effetti la pratica meditativa
raccomandata da Guan Yin, così come verrà in seguito descritta) indica
che la qualità centrale è piuttosto una recettività e un allinearsi con
le cose così come sono. Poi, a partire da questa accettazione radicale e
premurosa, tutte le mille mani di Guan Yin possono mettersi all'opera.

Le caratteristiche del Bodhisattva sono un simbolo spirituale che vuole
suggerire i diversi modi nei quali possiamo addestrare noi stessi.
Possiamo prendere la pratica dell'ascolto del suono interiore e usarla
per aiutarci a incarnare la compassione nelle nostre vite. Aprendo il
cuore per stare con il suono del silenzio e lasciando andare altre
preoccupazioni possiamo essere pienamente consapevoli e saggiamente
attenti al momento presente e a tutto ciò che contiene; attraverso
quell'attenta consapevolezza l'innata inclinazione compassionevole del
cuore puro viene risvegliata; quella attitudine compassionevole si
estende poi agli esseri intorno a noi. Inoltre il semplice addestramento
all'ascolto ha il suo impatto sul modo in cui ci relazioniamo con gli
altri. Si è spiegato prima come ascoltare il suono-nada ci aiuti ad
ascoltare i nostri pensieri. Ebbene, funziona altrettanto efficacemente
quando ascoltiamo gli altri. La gentilezza e la compassione implicano
entrambe molta pazienza e accettazione, e la pratica dell'ascolto è un
mezzo potente con il quale farle emergere e plasmarle. Ascoltare davvero
un'altra persona - senza reagire, senza entusiasmarsi, senza rifiutarla,
senza noia - è un'arte e una grazia. Aprirsi a ciò che sta dicendo e, in
tal modo, accoglierla completamente, è una benedizione sia per lei che
per noi stessi.

Su ancor più ampia scala possiamo estendere quest'attitudine di
attenzione compassionevole all'ascolto dei suoni del mondo, cosicché il
cuore si addestri ad abbracciare tutti gli esseri e i loro affanni. È
importante che questo non sia un abbraccio solo teorico, ma piuttosto -
proprio come Avalokiteśvara non ascolta solamente, ma ha molte teste,
mani e occhi, e mezzi abili - che l'intonarsi dei nostri cuori con il
mondo intero porti ad azioni e parole che aiutino nei modi più concreti
e tangibili. Nell'imparare a prestare attenzione al suono del silenzio
in questo modo, senza infatuazione, avversione o noia, stiamo
sviluppando un sentiero diretto verso quelle attitudini di gentilezza e
compassione che sono dimore sublimi per il cuore e illuminano di
splendore il mondo.

\section{Nada Aiuta a Penetrare Oltre L'Idea Di Un Sé}

Uno degli ostacoli principali a tali qualità sconfinate è un affidabile
piantagrane, l'idea di un sé. Fortunatamente possiamo usare il suono
interiore, nada, per sostenere lo sforzo di penetrare attraverso
quell'abitudine mentale creatrice di un sé, e la coazione a riprodurla
continuamente.

Una pratica che può aiutare a liberare il cuore da tali tendenze
compulsive è quella di meditare sul proprio nome. Iniziate soffermandovi
per qualche istante sul suono interiore. Focalizzatevi su di esso fino a
che la mente non sia chiara e aperta, vuota, poi pronunciate
semplicemente il vostro nome interiormente, qualunque esso sia. Per
iniziare ascoltate il suono del silenzio, fatto questo ascoltate il
suono del silenzio prima all'interno e poi sullo sfondo del vostro nome,
e infine il suono del silenzio dopo che avete ripetuto il nome,
``A-ma-ro'', ``Su-san'', ``John''. Osservate, sentite quali qualità
susciti quel suono. È soltanto il suono del vostro nome, così familiare,
così consueto per noi; osservate cosa accade quando viene calato nel
silenzio della mente e per una volta veramente sentito e conosciuto.
Osservate quali sono le qualità che arreca, come sblocca l'abitudine a
considerarci in un determinato modo: i confini si allentano. Con nostra
sorpresa, quel nome, quelle sillabe familiari, possono venire
improvvisamente percepite come la più peculiare e strana delle
formulazioni. Qualcosa nel cuore si risveglia e intuisce: ``Cos'ha a che
fare questo suono con qualcosa di reale?''. In quel momento ci rendiamo
conto che la parola che forma il nostro nome viene di solito riferita a
quella che in effetti è una qualità assolutamente non-personale.
Pronunciando così il nostro nome nel chiaro spazio aperto della saggezza
può apparirci come cercare di scriverlo con un fascio di luce su una
cascata. Non c'è nulla con cui fare un segno e nessuna superficie su cui
aderire.

Questo tipo di pratica può essere leggermente inquietante ma al contempo
gloriosamente liberante, e se le permettiamo davvero di liberarci, tutto
ciò che rimane è quel sapore di libertà, e il fragore della cascata.

\section{Nada e Investigazione}

Un altro modo forse persino più diretto per lavorare con l'ascolto
consiste nell'utilizzare una forma di investigazione per approcciare e
dissolvere abitudini di auto-referenzialità.

Ancora una volta ascoltate il suono del silenzio, focalizzatevi su di
esso per stabilizzare l'attenzione, rendete la mente il più possibile
silenziosa e vigile, e poi ponete la domanda: ``Chi sono io?''.

Prima ascoltate il suono del silenzio. Fatto questo ponete la domanda, e
poi siate presenti; notate cosa accade quando quella domanda viene posta
con sincerità: ''Chi sono io?''. Chiaramente non siamo alla ricerca di
una risposta verbale, concettuale. Notate invece che c'è uno spazio, un
breve spazio dopo che abbiamo posto la domanda e prima che appaiano le
risposte verbali, le risposte concettuali. Quando poniamo davvero quella
domanda: ``Chi sono io?'', oppure ``Cosa sono io?'', per un attimo si
spalanca un vuoto, uno spazio in cui il cuore si apre all'intuizione e
al dubbio circa le supposizioni costruite intorno all'essere una
persona: essere una donna, un uomo, vecchio, giovane. C'è un momento di:
``Oh!'', prima che tutti i dettagli personali comincino a insinuarsi.
C'è una pausa, un'esitazione. ``Chi sono io?''.

Fate sì che la vostra attenzione riposi in quell'intervallo dopo la fine
della domanda e prima che appaiano le risposte. Lasciate che la vostra
attenzione riposi in quell'intervallo, in quella spaziosità, perché, in
verità, il silenzio della mente è la risposta alla domanda. Permettete
alla mente di dimorare in quella spaziosità aperta, attenta, spontanea,
anzi incoraggiatela a farlo, perché in quel momento la concezione di un
proprio sé si interrompe. Le normali tendenze creatrici di un sé ne
risultano confuse, confutate, colte sul fatto. Improvvisamente, prima
che riescano a dileguarsi, l'obiettivo viene girato verso il fotografo.
È l'attimo decostruito, incondizionato. C'è attenzione. La mente è
vigile, tranquilla, luminosa. Ma non c'è alcun senso di un sé. È
straordinariamente semplice, naturale. Lasciate che l'attenzione dimori
in questo.

Dopo un po', quando appaiono altre più abituali ingerenze - un dolore
alla gamba, il suono di una macchina che passa, un prurito al naso -
quando le idee di un sé si sono nuovamente solidificate, allora
focalizzate con premura la mente, ritornate al suono-nada, ascoltate, e
ponete di nuovo la domanda: ``Chi sono io?'', per aprire quello stesso
spazio di curiosità, di realtà, per bucare la bolla
dell'auto-referenzialità anche solo per un attimo. Notate cosa accade
quando quella bolla non colora né distorce più la nostra visione delle
cose, e l'auto-costrutto viene meno. Cosa rimane? Come è la vita quando
quell'abitudine viene interrotta?

Come per la meditazione sul vostro nome, questa pratica può costituire
al tempo stesso una minaccia e un sollievo, ma se riusciamo a non farci
distrarre da nessuna delle due sensazioni e rimaniamo semplicemente
vigili e aperti al presente, ciò che realizziamo è la presenza di una
pura radiosità e pace, una radicale normalità, una benedetta semplicità,
il tutto avvolto nell'abbraccio di un fragoroso silenzio.

\section{Attributi Di Nada}

Diversi attributi del suono-nada incarnano utili qualità spirituali,
alcune delle quali lo rendono altrettanto universalmente accessibile e
valido quanto la consapevolezza del respiro, se non di più.

Innanzitutto il suono-nada come oggetto di meditazione incoraggia
un'attitudine di ascolto e ricettività, chiedendoci di essere
sperimentatori dal cuore aperto, piuttosto che organizzatori di
un'attività.

In secondo luogo, il suono non è soggetto a un controllo personale. A
differenza del respiro, che possiamo rendere più lungo o più corto, o
modificare in altri modi a nostro piacimento, non possiamo scegliere di
rendere il suono interiore più forte o più tenue, iniziarlo o
concluderlo, o in verità manipolarlo in alcun modo. Possiamo volgerci ad
esso e prestarvi attenzione oppure no, ma non è soggetto a intervento o
scelte personali. In questo modo incoraggia naturalmente la
realizzazione dell'assoluta impersonalità dell'esperienza, non avendo
alcuna caratteristica particolare che lo faccia intendere come ``me'' o
``mio''. Non è femminile o maschile, giovane o vecchio, intelligente o
stupido\ldots{} non ha dimensione o nazionalità, colore o lingua\ldots{}
semplicemente è, con l'imparzialità della Natura stessa.

Infine, è energizzante, ha una qualità naturalmente stimolante. Più gli
prestiamo attenzione, più tende a rendere luminosa la mente. Funziona in
un circuito di feedback positivo, cosicché quanto più forte è
l'attenzione su di esso, tanto più va ad alimentare la capacità di
essere attenti. In questo modo, aiutando la mente a essere più vigile, è
di supporto alla stessa pratica meditativa.

\section{Nada Come Simbolo Di Trascendenza}

Il suono del silenzio è un oggetto nella sfera dei sensi che riflette
molte caratteristiche del Dhamma come qualità trascendente, perciò può
esserne un'ottima presenza simbolica e un buon richiamo a quella Verità
Ultima.

Ad esempio, il suono-nada è sempre ``qui''. In questo modo è un buon
simbolo della qualità \emph{sanditthiko} del Dhamma, vale a dire essere
``manifesto qui e ora''.

È apparentemente senza inizio e senza fine, perciò ben rappresenta la
qualità \emph{akaliko} o atemporale del Dhamma. È impersonale, sempre
presente.

Una volta che vi abbiamo prestato attenzione, incoraggia
l'investigazione, perciò è in risonanza con l'attributo
\emph{ehipassiko} del Dhamma, l'``invito a venire a vedere''.

Conduce verso la dimensione interiore, scoraggiando l'assorbimento nel
mondo dei sensi, cosicché anche la qualità \emph{opanayiko} del Dhamma
vi è ben rappresentata.

Infine, richiede l'iniziativa di chi è interessato a prestargli
attenzione e dargli valore. In tal modo \emph{paccatam veditabbo
viññūhī}, l'``essere conosciuto dai saggi personalmente'', è
adeguatamente caratterizzato da quell'attributo.

Pertanto, pur essendo soltanto un semplice oggetto dei sensi, almeno
all'interno del sistema filosofico buddhista, i suoi attributi fanno sì
che si presti ad essere un buon simbolo del Dhamma stesso, un'eco, se
volete, nella sfera dei sensi, di quelle qualità fondamentali e
trascendenti della Verità Ultima.

\section{Nada e Le Sue Varie Manifestazioni}

Detto ciò, è vero che per alcune persone è molto difficile distinguere
questo suono interiore. Perciò, dopo aver letto tutto questo, vi starete
forse chiedendo: ``Ma di che accidenti sta parlando?''.

Non tutti riescono a cogliere facilmente quest'esperienza nella sfera
dell'ascolto. Può essere che a causa dei propri tratti caratteriali si
sia stati condizionati in modo diverso. Per esempio nel caso di un
grafico quella vibrazione interiore potrebbe essere più percepibile in
termini di qualità visiva, come una sottile oscillazione nel campo
visivo. Oppure, se qualcuno ha sviluppato molta consapevolezza del corpo
come insegnante di hatha yoga, potrebbe avvertirla nel corpo come una
qualità vibratoria delicata e pervasiva, un brusio echeggiante, un
formicolio alle mani, oppure come una presenza sottile ed energetica,
una continua corrente vitale che attraversa il corpo.

Spesso il modo in cui lo percepiamo dipende dal nostro condizionamento,
dalle nostre particolari abitudini e formazioni karmiche. Sulla base
della mia ventennale esperienza d'insegnamento di questo metodo ho
dedotto che per la maggior parte delle persone è più facilmente
percepibile nella sfera del suono. Per questo si parla di ``nada yoga'',
lo yoga, o la disciplina spirituale, del suono, ma se siete più capaci
di cogliere quella vibrazione universale attraverso la vista, il corpo,
o persino il tatto o l'olfatto, la pratica è altrettanto valida. La
focalizzazione sulla sua presenza e gli effetti che ne conseguono
funzionano esattamente nello stesso modo, indipendentemente dal mezzo
sensoriale attraverso cui la si esperisce. Può essere ugualmente usata
per tutte le pratiche sopra descritte e produrrà risultati equivalenti.

Se quella è la vostra propensione, è lì che troverete le ricompense più
ricche. Come dice l'adagio: ``L'oro è là dove lo trovi''.
