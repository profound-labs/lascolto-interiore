\begin{thebibliography}{9}

\bibitem{imaginary} \emph{An Imaginary Tale: The Story of $\sqrt{-1}$ (The
    square root of minus one)}, Princeton Library Science Edition por Paul J. Nahin

\bibitem{attention} \emph{The Law of Attention: Nada Yoga and the Way of Inner
    Vigilance} (Inner Traditions, seconda edizione, 2010) di Edward Salim Michael\\[2pt]
  {\fontsize{8}{8}\selectfont \url{http://www.innertraditions.com/the-law-of-attention.html}}

\bibitem{surangama} \emph{The Śūraṅgama Sūtra -- With Excerpts from the
    Commentary by the Venerable Master Hsüan Hua (Copyright 2009 di the
    Buddhist Text Translation Society)}

  Recentemente tradotta dal cinese dalla Commissione per la Traduzione dello
  Śūraṅgama Sūtra della Buddhist Text Translation Society: Rev. Bhikṣu Heng Sure
  (garante); Bhikṣu Jin Yan, Bhikṣu Jin Yong, Novizio Jin Jing, Novizio Jin Hai,
  Ron Epstein, David Rounds, Joey Wei, Fulin Chang, e Laura Lin

\bibitem{shabd} Da \emph{Naam or Word -- Book two: Shabd -- The Sound Principle}
  di Sant Kirpal Singh.

  {\footnotesize \url{http://www.ruhanisatsangusa.org/naam/naam-shabd1.htm}}

\bibitem{deep} Manoscritto non pubblicato. Cfr. \emph{Deep Listening}, un
  discorso di Dharma di Jan Chozen Bays

  {\footnotesize \url{http://www.zendust.org/audio/deep-listening}}
  
\bibitem{portrait} Dalla poesia ``\emph{Self Portrait}'' dell'autore, in
  ``\emph{Silent Rain}'', p. 263, e in ``\emph{Rain on the Nile}'', p. 65.

\bibitem{ensinamentos} Per esempio SN 55.5: ``Sāriputta, così è detto: `La
  corrente, la corrente'. Cosa è dunque, Sāriputta, la corrente?''.

  ``Questo Nobile Ottuplice Sentiero, venerabile signore, è la corrente; vale a
  dire: retta visione, retta intenzione, retta parola, retta azione, retti mezzi
  di sussistenza, retto sforzo, retta presenza mentale, retta concentrazione''.
  ``Molto bene, Sāriputta, molto bene! Questo Nobile Ottuplice Sentiero è la
  corrente''.

\end{thebibliography}




