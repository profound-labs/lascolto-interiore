\chapter{Citazioni}

\setlength{\parindent}{0pt}
\setlength{\parskip}{0.5\baselineskip}

\section*{Il Sūtra Sūrangama}

``Iniziai con una pratica basata sulla natura illuminata dell'ascolto.
Prima ho ridiretto il mio ascolto all'interno al fine di entrare nella
corrente dei saggi. Poi i suoni esterni sono spariti. Con la direzione
del mio ascolto rivolta all'incontrario e con i suoni acquietati, sia i
suoni che il silenzio cessarono di sorgere. Così accadde che, mentre
gradualmente progredivo, ciò che udivo e la mia consapevolezza di ciò
che udivo giunsero al termine. Anche quando quello stato mentale in cui
tutto era giunto al termine scomparve, non mi fermai. La mia
consapevolezza e gli oggetti della mia consapevolezza si svuotarono, e
quando quel processo di svuotamento della mia consapevolezza fu
interamente compiuto, allora persino quello svuotamento e ciò che era
stato svuotato svanirono. Nascita e dissolvimento cessarono anch'essi.
Allora la quiete ultima fu rivelata.

Tutto d'un tratto trascesi il mondo degli esseri ordinari, e trascesi
anche i mondi degli esseri che hanno trasceso i mondi ordinari. Ogni
cosa nelle dieci direzioni fu pienamente illuminata, e io ottenni due
poteri straordinari. In primo luogo, la mia mente si elevò per unirsi
con la mente sostanziale, meravigliosa, illuminata di tutti i Buddha in
tutte le dieci direzioni, e il mio potere di compassione divenne eguale
al loro. In secondo luogo, la mia mente discese per unirsi con tutti gli
esseri dei sei destini in tutte e dieci le direzioni cosicché provai le
loro sofferenze e le loro ardenti preghiere come fossero le mie''.

[\emph{pp. 234-5}]

\clearpage

\begin{quotation}
``Io ora dico rispettosamente questo a Colui che è Venerato in tutto
Mondo, Colui che divenne un Buddha in questo mondo di Sahā

Allo scopo di trasmetterci l'insegnamento autentico ed essenziale Inteso
per questo luogo, io dico che la purezza si trova tramite l'ascolto.
Tutti coloro che aspirano a conseguire la padronanza del samādhi
Troveranno certamente che l'ascolto è la via d'accesso''.

[\emph{p. 253}]
\end{quotation}

\begin{quotation}
``Grande Assemblea! Ānanda! Fermate lo spettacolo di burattini\\
Del vostro ascolto distorto! Capovolgete semplicemente il vostro udito\\
Per ascoltare la vostra vera e genuina natura,\\
La quale è la destinazione del Sentiero supremo.\\
Questo è il modo autentico per andare oltre, fino all'illuminazione''.

[\emph{p. 256}]
\end{quotation}

\smallskip

{\small\itshape
  Tratto da: ``The Śūraṅgama Sūtra''\\
  Recentemente tradotto dal cinese dalla Commissione per la Traduzione del
  Surangama Sutra della Buddhist Text Translation Society.
}

\clearpage

\section*{La Chāndogya Upaniṣad}

``La luce poi che risplende al di là del cielo, oltre ogni cosa, oltre
tutto, nei mondi supremi, insuperabili, in verità è quella stessa luce
che è dentro all'uomo\ldots{} L'ascolto di essa si ha quando, premendosi
le orecchie, si percepisce come un rumore, come il crepitare di un fuoco
che arde.''

[\emph{Up Ch 3.13.78}]

{\small\itshape
  Citato in ``Mind Like Fire Unbound'', cap. 1\\
  Traduzione inglese del Ven. Thanissaro Bhikkhu.\\
  Traduzione italiana di Carlo Della Casa in ``Induismo antico'', Mondadori,
  Milano, 2010
}

\section*{I Vagabondi Del Dharma}

Il silenzio è così intenso che riesci a sentire il rombo del tuo sangue
nelle orecchie ma molto più forte di questo suono è il rombo misterioso
che ho sempre identificato col rombare del diamante della saggezza, il
misterioso rombo del silenzio stesso, un grande Sssst che ricorda
qualcosa che ci sembra di avere dimenticato nella tensione dei nostri
giorni fin dalla nascita.

Desiderai poter esprimere tutto questo a coloro che amavo, a mia madre,
a Japhy, ma non c'erano parole per esprimerne la vacuità e la purezza.
``C'è forse un insegnamento indiscutibile e chiaro da impartire a tutte
le creature viventi?'' era la domanda che probabilmente qualcuno fece al
nevoso Dipankara dalle sopracciglia folte, e la sua risposta fu il rombo
di silenzio del diamante.

{\small\itshape
  ``I Vagabondi Del Dharma''. Cap.22, di Jack Kerouac.\\
  Traduzione di Nicoletta Vallorani, Mondadori, Milano, 2007
}

\clearpage

\section*{La Salita Del Monte Carmelo}

(Subida Del Monte Carmelo)

Niente, niente, niente, niente, niente.\\
E anche sul Monte, niente.\\
(Nada, nada, nada, nada, nada. Y en el Monte, nada.)

{\small\itshape
  ``La Salita del Monte Carmelo -- La Via Del Puro Spirito''.\\
  di San Giovanni della Croce
}

\setlength{\parskip}{0pt}
\setlength{\parindent}{17pt}
